\documentclass[]{report}
\usepackage[margin=1in]{geometry}
\usepackage{parskip}
\title{What is cloud computing?}
\author{George Lancaster - qv18258}
\begin{document}
\begin{centering}
\section*{\underline{  Cloud Computing - FA1  }}
\end{centering}

Cloud computing is an umbrella term for on-demand computing services provided over the internet. To be considered a cloud service, computing resources must be elastically allocated and de-allocated to scale with demand. Resource allocation is metered, and provided to the user at a charge \cite{mell2011nist}. Consequently, cloud based applications can be developed on any scale with little to no investment in hardware. Cloud computing is also known as `utility computing', referencing how the business model is similar to utilities like gas, electricity and water. 
\par
The three most common types of cloud computing services are: Software as a Service (SaaS), Platform as a service (PaaS) and Infrastructure as a Service (Iaas) \cite{cliff2010remotely}. SaaS is a software distribution model whereby a user accesses a software package exclusively through a thin client, in most cases a web-browser. The back-end of the program is stored remotely on a server, which processes the client's requests much like a client-server application \cite{turner2003turning}. PaaS provides a platform for the development and deployment of web-based applications. The software is provided as a pre-installed stack on the cloud machines, on which developers can produce SaaS applications. IaaS provides cloud services at the lowest level of abstraction. They allow complete customisation of the development environment on the cloud machines, including a choice of operating system.
\par
Cloud services can be categorised as public, private or hybrid. A public cloud service is operated by a third-party company, and all computing resources are shared by all users of the cloud. Upon the deallocation of a cloud resource, the data from the previous session is deleted, so that it can be reallocated to a new user \cite{ren2012security}. A private cloud restricts its usage to a single company. They can be operated internally, or by a third party provider. They are more commonly used by organisations like government agencies, who require greater control and security over the data they store. A hybrid cloud uses both private and public solutions. It does this by maintaining its own private cloud, but using a public service when required. An example of this being `cloud-bursting', which enables a service to run in the private cloud until there is an increase in usage, requiring the use of public resources \cite{guo2014cost} \cite{WinNT}. 
\par
Perhaps the most successful cloud computing provider is Amazon. In 2006, Amazon re-released Amazon Web Services (AWS) as the first major cloud service provider. AWS currently offers both PaaS and IaaS services to its users \cite{carr2008big}. AWS was made possible by a large investment in computing infrastructure by Amazon, which allowed the company to rent unused servers to their customers.
\par
The popularity of cloud computing can be attributed to its innate ease of implementation and scalability. Before the existence of the cloud, a company wanting to set up a web-service would have to invest a large amount of capital expenditure on hardware and land for hosting. Services like AWS have eliminated the need to do this, and therefore lowered the barrier to entry in providing web-services. 

\newpage
\bibliography{citations}
\bibliographystyle{apalike}



\end{document}
